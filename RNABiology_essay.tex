\documentclass[a4paper, twocolumn]{article}
\usepackage[numbers,sort&compress]{natbib}
\renewcommand{\bibnumfmt}[1]{#1.}
\usepackage[english]{babel}
\usepackage[utf8]{inputenc}
\usepackage{amsmath}
\usepackage{url}
\usepackage{graphicx}
\usepackage[colorinlistoftodos]{todonotes}

\title{RNA Polymerases IV and V: evolution and function}

\author{Carlotta Porcelli, qbp693}

\date{\today}

\begin{document}
\maketitle

\begin{abstract}
	
Pol-IV, which functions to initiate siRNA biogenesis, and Pol-V, which functions to generate scaffold transcripts that recruit downstream RdDM factors. 

\end{abstract}

\section{Introduction}
Most of the known non-coding RNAs (ncRNAs) correspond to intergenic sequences or transcripts with unknown functions. In mammals, two of the increasingly known long ncRNAs are \textit{Xist} and \textit{Tsix} which are involved in the regulation of adjacent genes. 
In plants, small-interfering RNAs (siRNAs) guide the process of chromatin modifications. These ncRNAs are generated from dsRNAs precursors and processed by dicer (DCL) enzymes into 21-24 nucleotides long siRNAs. Their association with Argonaute proteins (AGO) guide chromatin modification to homologous DNA sequences \cite{BRODERSEN2006}.


\section{The paradox of epigenetic control}
Need for transcription in order to transcriptionally silence the same region. \cite{paradox}

\subsection{The RNA-mediated DNA-methylation pathway (RdDM)s}


\section{RNA Polymerase IV}
how it produces siRNAs and what do they bind to : AGO specific and dicer specific. 
from:  \cite{Zhang130} and from \cite{ONODERA2005}
picture from: \cite{Xu2013}

\subsection{The subunits}

\begin{itemize}
	\item NRPD

\end{itemize}

\subsection{The products and their function}


\begin{itemize}
	\item what genes are silenced and how
	\item DNA methylation 
\end{itemize}

\cite{PIKAARD2008}


\section{RNA Polymerase V}
Nuclear RNA polymerase V (Pol V) is a multi-subunit plant-specific RNA polymerase with a role in the siRNA-directed DNA methylation (RdDM) pathway and transcriptional gene silencing. \cite{ZHOU2015}.


The siRNAs produced by Pol IV associated with AGO4 or AGO6 proteins guide cytosine methylation through base-pairing interaction with the nascent transcripts of Pol V. \cite{Wierzbicki2009}.

lncRNAs (long non-coding RNAs) transcripts


see: \cite{Daxinger_2008}

see: \cite{wierzbick1}

\subsection{The subunits}
\cite{Wendte2017}
\cite{ZHOU2015}
\begin{itemize}
		\item NRPE
\end{itemize}


\cite{REAM2009}

\subsection{The function}




As shown in \cite{LAHMY2016} in the C-terminal domain (CTD) of Pol V largest subunit NRPE1, and its associated factor SPT5L (a Pol V auxiliary protein), is present a conserved domain of glycine-tryptophan/tryptophan-glycine (GW/WG) region. Its function is shown to be essential to hook AGO proteins and facilitate the bond of their siRNA sequences to the DNA chromatin targets. These motifs recruit a massive amount of AGO4 siRNA loaded complexes to their site of action. The increase of local concentration of AGO4 is a consequence of the stabilization of Pol V to the DNA strand. The model proposed sees Pol V proceeding through transcription elongation, AGO4 interacting with the lncRNAs Pol V-transcripts generating an intermediate complex that would serve for the transfer of AGO4 to the DNA template. At this stage AGO4 recruits DRM2 (DNA (cytosine-5)-methyltransferase) to the opposite DNA strand to trigger DNA methylation.



\section{Future outlook}

It is still unclear what templates the two polymerases use \textit{in vivo} and how the transcription is regulated in order not to produce aberrant non-coding RNAs. 

Further studies could be led to investigate both the transcription initiation and the role of the RNA primers in that and the termination steps. 


\subsection{On Pol IV}

The recruitment of the Pol IV to targets different from H3K9 methylated sites is still unclear. Moreover the SHH1 factor, needed for Pol IV association with chromatin, recognizes H3K9 methylation and unmethylated H3K4 on the loci where the Pol IV downstream is going to be recruited. \cite{LAW2013} It is still not clear how the histone methylation are initially established in the specific loci.


\subsection{On Pol V}

The association of Pol V to the chromatin involves seems to involve more factors than the few one discussed in this work, as shown in \cite{STROUD2012}. Nevertheless  their specific function is still unclear and could be elucidated in further studies. 


\bibliography{essay_bibliography}
\bibliographystyle{currbiol}

\end{document}