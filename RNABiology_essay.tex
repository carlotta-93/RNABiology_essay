\documentclass[a4paper, twocolumn]{article}
\usepackage[numbers,sort&compress]{natbib}
\renewcommand{\bibnumfmt}[1]{#1.}
\usepackage[english]{babel}
\usepackage[utf8]{inputenc}
\usepackage{amsmath}
\usepackage{graphicx}
\usepackage[colorinlistoftodos]{todonotes}

\title{RNA Polymerases IV and V: evolution and function}

\author{Carlotta Porcelli, qbp693}

\date{\today}

\begin{document}
\maketitle

\begin{abstract}
This is gonna be the last thing i write down

\end{abstract}

\section{Introduction}
Most of the known non-coding RNAs (ncRNAs) correspond to intergenic sequences or transcripts with unknown functions. In mammals, two of the increasingly known long ncRNAs are \textit{Xist} and \textit{Tsix} which are involved in the regulation of adjacent genes. 
In plants, small-interferingRNAs (siRNAs) guide the process of chromatin modifications. These ncRNAs are generated from dsRNAs precursors and processed by dicer (DCL) enzymes into 21-24 nucleotides long siRNAs before associating with Argonaute proteins (AGO). 

\section{RNA Polymerase IV}
how it produces siRNAs and what do they bind to : AGO specific and dicer specific \cite{Zhang130}

\subsection{Subunits}

\subsection{The products}

\subsection{The function}

\section{RNA Polymerase V}

\subsection{Subunits}

\subsection{The products}

\subsection{The function}

\section{The paradox of epigenetic control}
Need for transcription in order to transcriptionally silence the same region. \cite{paradox}

\section{Future outlook}


\bibliography{essay_bibliography}
\bibliographystyle{currbiol}

\end{document}