\documentclass[a4paper, twocolumn]{article}
\usepackage[numbers,sort&compress]{natbib}
\renewcommand{\bibnumfmt}[1]{#1.}
\usepackage[english]{babel}
\usepackage[utf8]{inputenc}
\usepackage{amsmath}
\usepackage{url}
\usepackage{graphicx}
\usepackage[colorinlistoftodos]{todonotes}

\title{RNA Polymerases IV and V: evolution and function}

\author{Carlotta Porcelli, qbp693}

\date{\today}

\begin{document}
\maketitle

\begin{abstract}
	
Pol-IV, which functions to initiate siRNA biogenesis, and Pol-V, which functions to generate scaffold transcripts that recruit downstream RdDM factors. 

\end{abstract}

\section{Introduction}
Most of the known non-coding RNAs (ncRNAs) correspond to intergenic sequences or transcripts with unknown functions. In mammals, two of the increasingly known long ncRNAs are \textit{Xist} and \textit{Tsix} which are involved in the regulation of adjacent genes. 
In plants, small-interfering RNAs (siRNAs) guide the process of chromatin modifications. These ncRNAs are generated from dsRNAs precursors and processed by dicer (DCL) enzymes into 21-24 nucleotides long siRNAs before associating with Argonaute proteins (AGO). 

\section{The paradox of epigenetic control}
Need for transcription in order to transcriptionally silence the same region. \cite{paradox}

\subsection{The RNA-mediated DNA-methylation pathway (RdDM)s}


\section{RNA Polymerase IV}
how it produces siRNAs and what do they bind to : AGO specific and dicer specific. 
from:  \cite{Zhang130} and from \cite{ONODERA2005}
picture from: \cite{Xu2013}

\subsection{The subunits}

\begin{itemize}
	\item NRPD

\end{itemize}

\subsection{The products and their function}


\begin{itemize}
	\item what genes are silenced and how
	\item DNA methylation 
\end{itemize}

\cite{PIKAARD2008}


\section{RNA Polymerase V}
* open up dna or make transcripts for binding for ago ? 
see: \cite{Daxinger_2008}

see: \cite{wierzbick1}

\subsection{The subunits}
\cite{Wendte2017}
\cite{subunits}
\begin{itemize}
		\item NRPE
\end{itemize}

\cite{REAM2009}

\subsection{The function}

two proposed models as in \cite{PIKAARD2008}




\section{Future outlook}

It is still unclear what templates the two polymerases use \textit{in vivo} and how the transcription is regulated in order not to produce aberrant non-coding RNAs. 

Further studies could be led to investigate both the transcription initiation and the role of the RNA primers in that and the termination steps. 


\subsection{On Pol IV}

The recruitment of the Pol IV to targets different from H3K9 methylated sites is still unclear. Moreover the SHH1 factor, needed for Pol IV association with chromatin, recognizes H3K9 methylation and unmethylated H3K4 on the loci where the Pol IV downstream is going to be recruited. \cite{LAW2013} It is still not clear how the histone methylation are initially established in the specific loci.


\subsection{On Pol V}

The association of Pol V to the chromatin involves seems to involve more factors than the few one discussed in this work, as shown in \cite{STROUD2012}. Nevertheless  their specific function is still unclear and could be elucidated in further studies. 


\bibliography{essay_bibliography}
\bibliographystyle{currbiol}

\end{document}